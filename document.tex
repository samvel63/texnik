\documentclass[10pt]{book}
\usepackage[a5paper,top=54pt,bottom=54pt,left=48pt,right=48pt]{geometry}
\usepackage[utf8]{inputenc}
\usepackage[T1,T2A]{fontenc}
\usepackage[english,russian]{babel}
\usepackage{graphicx}
\usepackage{amsmath,amsthm,amssymb}
\usepackage{caption2}
\usepackage{yhmath}
\linespread{0.5}

%page header
\usepackage{fancybox,fancyhdr}
\pagestyle{fancy}
\fancyhead{}
\fancyhead[LE,RO]{\textbf{\thepage}}
\fancyhead[RE]{\textit{\textsection 47. Криволинейные интегралы}}
\fancyhead[LO]{\textit{47.8 Интегралы, не зависящие от пути интегрирования}}
\fancyfoot{}
\renewcommand{\headrulewidth}{0pt}

\setcounter{page}{138}
\setcounter{figure}{149}

%remove colon after "Рис. %number%"
\renewcommand{\captionlabeldelim}{~}

%font
\fontfamily{lh}
\selectfont

\usepackage{pgfpages}
\pgfpagesuselayout{2 on 1}[a4paper,landscape,border shrink=5pt]

\begin{document}
	
	\noindent отображение $F$ мы наложили несколько более сильные условия,\linebreak
	потребовав непрерывности смешанных производных $\dfrac{d^2y}{du \ dv}$ и $\dfrac{d^2y}{dv \ du}$ и\linebreak
	возможности применения формулы Грина для области {Г*}). Нетрудно \linebreak
	убедиться и в том, что стремление к пределу в формуле (47.27) \linebreak
	происходит равномерно в смысле, указанном в теореме 1 п. 46.1 \par
	
	Несмотря на простоту вывода формулы (47.27), следует отметить, \linebreak
		что доказательство теоремы 1, приведенное в п. 46.1, идейно пред- \linebreak
		почтительнее, так как оно лучше раскрывает сущность вопроса, \linebreak
		связанную с тем, что дифференцируемое отображение в малом до- \linebreak
		статочно хорошо аппроксимируется линейным отображением. \par
	\begin{center}
		{\textbf{47.8. Криволинейные интегралы, не зависящие \linebreak от пути интегрирования}}
	\end{center}	
	
	
	Все кривые (контуры), рассматриваемые в этом пункте,\linebreak
	будут всегда предполагаться кусочно-гладкими; для краткости это\linebreak
	не будет каждый раз специально оговариваться.\par
	
	Рассмотрим вопрос о том, когда криволинейным интеграл \linebreak
	$\int\limits_{\wideparen{AB}} P dx + Q dy$ зависит только от точек A и B и не зависит от выбора \linebreak
	кривой $\wideparen{AB}$, их соединяющей. \par 
	
	\textit{\textbf{Теорема 3.} Пусть функции P(x, y) и Q(x, y) непрерывны в пло-\linebreak 
	ской области G, тогда эквивалентны следующие три условия.} \par
	
		
	 \textit{1. Для любого замкнутого контура $\gamma$, лежащего в G,} \par 
	
	$$ \int\limits_{\gamma} P dx + Q dy = 0.$$
	
	 \textit{2. Для любых двух точек A  $\in$ G и B $\in$ G значение интеграла} \linebreak 
	
	$$\int\limits_{\wideparen{AB}} P dx + Q dy$$
	
	\textit{не зависит от кривой $\wideparen{AB} \subset G$, соединяющей точки A и B.} \par 
	\textit{3. Выражение P dx + Q dy является в G полным дифференциалом,\linebreak
	т. е. существует функция u(M) = u(x, y), M = (x, y), определен-\linebreak
	ная в G и такая, что} \par
	
	$$du=P dx + Q dy.$$
	\textit{В этом случае если A  $\in$ G и B $\in$ G, то}\par 
	
	$$ \int\limits_{\wideparen{AB}} P dx + Q dy = u \left( B \right) - u \left( A \right) $$
	\textit{для любой кривой $\wideparen{AB}$, соединяющей в G эти точки.} \par 
	\textit{Таким образом, выполнение каждого из условий 1, 2 и 3 необходимо \linebreak и достаточно для выполнения каждого из двух остальных.} \par
	
	Доказательство. Покажем, что из первого условия сле- \linebreak
	дует второе
	
\end{document}